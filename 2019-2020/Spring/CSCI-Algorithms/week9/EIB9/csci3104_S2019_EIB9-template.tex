\documentclass[12pt]{article}
\setlength{\oddsidemargin}{0in}
\setlength{\evensidemargin}{0in}
\setlength{\textwidth}{6.5in}
\setlength{\parindent}{0in}
\setlength{\parskip}{\baselineskip}
\usepackage{amsmath,amsfonts,amssymb}
\usepackage{graphicx}
\usepackage[]{algorithmicx}

\usepackage{fancyhdr}
\pagestyle{fancy}

%\usepackage{hyperref}


\setlength{\headsep}{36pt}

\begin{document}

\lhead{{\bf CSCI 3104, Algorithms \\ Explain-It-Back 9} }
\rhead{Name: \fbox{Keaton Whitehead} \\ ID: \fbox{104668369} \\ {\bf Profs.\ Grochow \& Layer\\ Spring 2019, CU-Boulder}}
\renewcommand{\headrulewidth}{0.5pt}


.

Hey Jill!

	Get ready to buy a boat and the 5 mansions around the world that you've always wanted!
	
I was working on developing software that will help automate your buy and sell orders that you receive when I noticed a loop hole where we can rack in the dough without having to do anything!
So based off of the information that you gave me I noticed that if we begin with 100 pounds in GBD currency then transfer it to CAD we get a total of 175 Canadian dollars. If we then exchange that to USD we end up with 131.25 and then if we exchange it back to GBD we end with 101.0625. We end up with 1.0625 more currency then we first started off with! I was thinking that we could set up a program that automatically pushes our money through this exchange loop and we would automatically grow whatever investment and it will grow at a rate of 1.010625 percent every time we get our money back. So if say we invest 1,000,000 pounds we will get a return of 1010625 pounds, which is a 10,625 pound profit! But that's when I realized that we are only looking at these three currencies. We should incorporate all of the currencies because there could be an exchange path that results in a higher return percentage. Even the smallest increase from 1.010625 will result in massive profit because if we use the same 1,000,000 British pounds and find an exchange path that is let us say 1.02 percent increase we will get a return of 20,000 pounds. A small increase of just .009375 percent resulted in an extra profit of 9375 pounds. So these small percentages really do matter! Also we can have this run continually every second, or minute depending on the time it takes to do the exchange, to maximize how many times we can use this exchange loop.

	The best part too is that I know of an algorithm that can easily calculate the path that can maximize our profits. It's called the Bellman-Ford method. This can search through all of the currencies and their exchange rates and find the shortest exchange path with the highest profits efficiently. So if  
we maximize the exchange path, we maximize the profits we can make, and finding the shortest path allows us to do the exchange even faster so we are able to perform more exchanges because our exchange comes back faster.
So I apologize that I am going to stop working on the original assignment you gave me but I think within a few days I will no longer need to work another day in my life, and don't worry I will let you use this so you don't need to work any long :). I hope you are doing well and if not this will deffinetly change your life once I get done with it! 

Anyways I'll see you soon! - Keaton Whitehead
\mbox{}
\newpage
\pagebreak
\end{document}
