\documentclass[12pt]{article}
\setlength{\oddsidemargin}{0in}
\setlength{\evensidemargin}{0in}
\setlength{\textwidth}{6.5in}
\setlength{\parindent}{0in}
\setlength{\parskip}{\baselineskip}
\usepackage[all]{xy}

\usepackage{amsmath,amsfonts,amssymb}
\usepackage{graphicx}
\usepackage{fancyhdr}
\pagestyle{fancy}
\usepackage{hyperref}


\begin{document}

\lhead{{\bf CSCI 3104 \\ Problem Set 10} }
\rhead{Name: \fbox{\phantom{This is a really long name}} \\ ID: \fbox{\phantom{This is a student ID}} \\ {\bf Profs.\ Grochow \& Layer\\ Spring 2019, CU-Boulder}}
\renewcommand{\headrulewidth}{0.5pt}
\phantom{Test}

Quick links \ref{1a} \ref{1b} \ref{1c} \ref{2a} \ref{2b} \ref{2c} \ref{3a} \ref{3b}

\vspace{-3mm}
\begin{enumerate}

\item A \emph{matching} in a graph $G$ is a subset $E_M
    \subseteq E(G)$ of edges such that each vertex touches at most one of the
        edges in $E_M$. Recall that a bipartite graph is a graph $G$ on two
        sets of vertices, $V_1$ and $V_2$, such that every edge has one
        endpoint in $V_1$ and one endpoint in $V_2$. We sometimes write $G =
        (V_1, V_2; E)$ for this situation. For example:
	
	\[
\xymatrix{
V_1: & 1 \cir{}  \ar@{-}[dr] & 2 \ar@{-}[dl] \ar@{.}[d] \ar@{.}[dr] & 3 \ar@{.}[dll] \ar@{-}[dr]\ar@{.}[drr]  & 4 \ar@{.}[dll] \ar@{-}[dl] \ar@{.}[d] & 5 \ar@{.}[dll]  \ar@{.}[d]  & 6 \ar@{-}[dl] \\
V_2: & 7 & 8 & 9  & 10 & 11 
}
\]

The edges in the above example consist of all the lines, whether solid or dotted; the solid lines form a matching. 

The \emph{bipartite maximum matching} problem is to find a matching in a given bipartite graph $G$ , which has the maximum number of edges among all matchings in $G$.
\pagebreak

\begin{enumerate}
    \item (6 pts total)  \label{1a} Prove that a maximum matching in a bipartite graph $G = (V_1, V_2; E)$ has size at most $\min\{|V_1|, |V_2|\}$.
% YOUR SOLUTION HERE
\pagebreak

\item (8 pts total) \label{1b} Show how you can use an algorithm for max-flow to solve bipartite maximum
    matching on undirected simple bipartite graphs. That is, give an algorithm
        which, given an undirected simple bipartite graph $G = (V_1, V_2; E)$,
        (1) constructs a directed, weighted graph $G'$ (which need not be
        bipartite) with weights $w:E(G') \to \mathbb{R}$ as well as two
        vertices $s,t \in V(G')$, (2) solves max-flow for $(G', w),s,t$, and
        (3) uses the solution for max-flow to find the maximum matching in $G$.
        Your algorithm may use any \texttt{max-flow} algorithm as a subroutine.
% YOUR SOLUTION HERE
\pagebreak

\item (7 pts total) \label{1c} Show the weighted graph constructed by your algorithm on the example bipartite graph above.
% YOUR SOLUTION HERE
\pagebreak

\end{enumerate}
	
\item In the review session for his Deep Wizarding class,
    Dumbledore reminds everyone that the logical definition of NP requires that
        the number of \textit{bits} in the witness $w$ is polynomial in the
        number of bits of the input $n$. That is, $|w| = poly(n)$. With a
        smile, he says that in beginner wizarding, witnesses are usually only
        logarithmic in size, i.e., $|w| = O(\log n)$.  \begin{enumerate} 
\pagebreak
\item (7 pts total)  \label{2a} Because you are a model student, Dumbledore asks you to prove, in front
    of the whole class, that any such property is in the complexity class P.
% YOUR SOLUTION HERE
\pagebreak

\item (6 pts total) \label{2b} Well done, Dumbledore says. Now, explain why the logical definition of NP
    implies that any problem in NP can be solved by an exponential-time
    algorithm.
% YOUR SOLUTION HERE
\pagebreak

\item (6 pts total) \label{2c} Dumbledore then asks the class: ``So, is NP a good formalization of the
    notion of problems that can be solved by brute force?  Discuss.'' Give
    arguments for both possible answers.
% YOUR SOLUTION HERE
\pagebreak

\end{enumerate}

\item (20 pts ) Recall that the \textit{MergeSort} algorithm is a sorting
algorithm that takes $\Theta(n\log n)$ time and $\Theta(n)$ space. In this
problem, you will implement and instrument MergeSort, then perform a
numerical experiment that verifies this asymptotic analysis. There are two
functions and one experiment to do this.
	
        (i) {\tt MergeSort(A,n)} takes as input an unordered array $A$, of
        length $n$, and returns both an in-place sorted version of $A$ and a
        count $t$ of the number of atomic operations performed by {\tt
        MergeSort}.
	
        (ii) {\tt randomArray(n)} takes as input an integer $n$ and returns an
        array $A$ such that for each $0 \leq i < n$, $A[i]$ is a uniformly
        random integer between $1$ and $n$. (It is okay if $A$ is a random
        permutation of the first $n$ positive integers.)
\pagebreak
		
	\begin{enumerate}
            \item (10 pts total) \label{3a} From scratch, implement the functions {\tt MergeSort} and {\tt
            randomArray}. You may not use any library functions that make their
                implementation trivial. You may use a library function that
                implements a pseudorandom number generator in order to
                implement {\tt randomArray}. 
	
        Submit a paragraph that explains how you instrumented {\tt MergeSort},
                i.e., explain which operations you counted and why these are
                the correct ones to count.
% YOUR SOLUTION HERE
\pagebreak
	
        \item (10 pts total)  \label{3b} For each of $n= \{2^{4},2^{5},\dots,2^{26},2^{27}\}$, run 
            {\tt MergeSort(randomArray(n),n)} fives times and record the tuple
            $(n,\langle t\rangle)$, where $\langle t\rangle$ is the average
            number of operations your function counted over the five
            repetitions. Use whatever software you like to make a line plot
            of these 24 data points; overlay on your data a function of the
            form $T(n) = A\,n \log n$, where you choose the constant $A$ so
            that the function is close to your data.
	
	Hint 1: To increase the aesthetics, use a log-log plot.\\
        Hint 2: Make sure that your \textit{MergeSort} implementation uses only
                two arrays of length $n$ to do its work. (For instance, don't
                do recursion with pass-by-value.)
% YOUR SOLUTION HERE
\pagebreak
	
	\end{enumerate}
	

\end{enumerate}


\end{document}


