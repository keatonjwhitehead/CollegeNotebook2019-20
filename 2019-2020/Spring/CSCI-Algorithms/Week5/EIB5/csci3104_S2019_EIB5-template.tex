\documentclass[12pt]{article}
\setlength{\oddsidemargin}{0in}
\setlength{\evensidemargin}{0in}
\setlength{\textwidth}{6.5in}
\setlength{\parindent}{0in}
\setlength{\parskip}{\baselineskip}
\usepackage{amsmath,amsfonts,amssymb}
\usepackage{graphicx}
\usepackage[]{algorithmicx}

\usepackage{fancyhdr}
\pagestyle{fancy}

%\usepackage{hyperref}


\setlength{\headsep}{36pt}

\begin{document}

\lhead{{\bf CSCI 3104, Algorithms \\ Explain-It-Back 5} }
\rhead{Name: \fbox{Keaton Whitehead} \\ ID: \fbox{104668391} \\ {\bf Profs.\ Grochow \& Layer\\ Spring 2019, CU-Boulder}}
\renewcommand{\headrulewidth}{0.5pt}

\phantom{Test}

At a recent infections disease seminar, you hear about how the speaker is
sequencing millions of {\sl Plasmodium falciparum} genomes (the human malaria
parasite) in order to better characterize how patients respond to treatment. In
the presentation, the speaker complained to the audience that although they are
making the data sets a small as possible by only using 2 bits to encode the 4
nucleotides of the genomes their IT department is still struggling to store all
the data. Later in the presentation, you learn that the Plasmodium falciparum
genome is "AT-rich." That is, over 80\% of the nucleotides in the genome are
either A or T. Please help this team understand how they can leverage
Plasmodium falciparum’s AT-richness to help their IT department deal with the
influx of data.

\pagebreak

\newpage
\mbox{}
\newpage
\pagebreak
\end{document}


